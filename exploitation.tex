\documentclass{article}
\usepackage[utf8]{inputenc}
\usepackage[margin=0.3in]{geometry}

\begin{document}
During the day, workers perform labor of all kinds, and this labor can be divided up based on the purpose it serves. First, there is the labor done to produces all of the goods and services used by the workers themselves on a regular basis. The total amount of time taken each production period to prouce these particular goods/services is denoted $V$. Additionally, there is the labor done which is in excess both of this labor which is `for the workers' and of the labor required to maintain and reproduce the means of production. This `surplus labor' goes entirely to the capitalist class as tribute, one way or another. The total amount of labor time which falls in this category is denoted $S$. \\ \par The ratio of these two numbers, $\frac{S}{V}$, denoted $e$ and called $\textbf{the rate of exploitation}$. Uh can thus be used a measure of the class struggle itself. A larger $e$ indicates a greater portion of the social labor going to the capitalists, whereas a smaller $e$ indicates a greater portion going to the workers. \\ \par It is worth noting that $e$ is a global property of the society. It is not an average, but rather an aggregate quantity which is looking at the entire working class as a unity against the entire capitalist class as a unity. One worker in this society might be worked harder or more unfairly than another; the rate of exploitation says nothing about how much alienation the individual worker is feeling, or how exploited they feel they are by their boss. Rather, it can perhaps be seen more accurately as a measure of how much \textit{social} alienation the average worker feels \textit{as a member of their society}. This includes, for example, the alienation that one feels in response to their tax dollars being used to subsidize the profits of a ruling class, or in response to being taken advantage of by a for-profit healthcare system.
\end{document}
